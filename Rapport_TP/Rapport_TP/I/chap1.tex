\chapter{Étude d'un système de transport}
Dans cette première partie, nous allons vous présenter nos travaux sur l'étude d'un système de transport. Dans ce problème, il nous est demandé de mettre en œuvre, avec l'algèbre $(max,+)$, une étude d'un système de GET qui représentera notre système temporisé de transport. Nous commencerons par vous présenter le graphe d'évènements correspondant au procédé étudié, nous effectuerons une analyse de ce système pour enfin essayer d'améliorer le système en modifiant le graphe.

\section{Graphe d'évènements temporisé}

\section{Analyse, matrice d'évolution et valeurs propres}
\begin{align*}
\left\lbrace
\begin{array}{lcl}
x_{A1}(k) &=&7x_{B1}(k)\\ 
x_{B1}(k) &=&6x_{C1}(k)\\
x_{D1}(k) &=&2x_{D2}(k-1)\\
x_{D2}(k) &=&4x_{C2}(k)\\
x_{C2}(k) &=&6x_{B1}(k)\\
x_{B2}(k) &=&7x_{A2}(k) \oplus x_{B12}\\
x_{A2}(k) &=&2x_{A1}(k-2)\\
x_{A22}(k) &=&x_{A1}(k-1)\\
x_{E1}(k) &=&2x_{E2}(k-1)\\
x_{F1}(k) &=&5x_{E1}(k)\\
x_{B12}(k) &=&3x_{F1}(k)\\
x_{G1}(k) &=&2x_{B12}(k)\\
x_{G2}(k) &=&2x_{G1}(k-1)\\
x_{B22}(k) &=&2x_{G2}(k) \oplus x_{B2}(k)\\
x_{F2}(k) &=&3x_{B22}(k)\\
x_{E1}(k) &=&5x_{F2}(k)   \\  
\end{array}
\right.
\end{align*}
lala
\section{Commande des conditions initiales}


