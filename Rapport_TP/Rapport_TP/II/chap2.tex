\chapter{Optimisation d'une ressource dans un problème de jobshop}
Cette fois ci, nous allons nous concentrer sur la modélisation d'un problème à base de \emph{jobshop} dans lequel nous étudierons le travail entre deux machines $M_1$ et $M_2$ sur deux pièces $A$ et $B$. Dans un premier temps, nous modéliserons comme dans le chapitre précédent le GET associé à notre système. Puis nous étudierons chaque marquage initial de notre graphe, selon les ordonnancement possible, pour déterminer le cycle associé de chacun.

\section{GET associé à chaque solution}
%% Ordonnancement 1	:		M2 --> A		B
%%							M1 -->   A - B

%% Ordonnancement 2	:		M2 --> A		B	NON RESPECT
%%							M1 -->   A - B		NON RESPECT

%% Ordonnancement 3	:		M2 --> A - B
%%							M1 --> B - A

%% Ordonnancement 4	:		M2 --> 	  B	- A
%%							M1 --> B 		A

\section{Représentation d'état et analyse du cycle associé}
\begin{align*}\label{equ:ordo1}
\left\lbrace
\begin{array}{lcl}
x_{AM_1}(k)&=& 10x_{AM_1}(k-1) \oplus 9x_{BM_2}(k-1)\oplus x_{BM_1}(k-1)\\
x_{AM_2}(k)&=&4x_{AM_1}(k-1) \oplus 3x_{BM_2}(k-1)\\
x_{BM_1}(k)&=& 11x_{BM_1}(k-1) \oplus 14x_{AM_1}(k-1) \oplus 13x_{BM_2}(k-1)\\
x_{BM_2}(k)&=& 20x_{BM_2}(k-1) \oplus 21x_{AM_1}(k-1) \oplus 18x_{BM_1}(k-1)\\
\end{array}
\right.
\end{align*} qui donne une représentation d'état suivante : 
\begin{align*}
X(k)
\end{align*}
L'ordonnancement 2 n'est pas réalisable.
\begin{align*}%\label{equ:ordo3}
\left\lbrace
\begin{array}{lcl}
x_{AM_1}(k)&=& 10x_{AM_1}(k-1) \oplus  9x_{BM_2}(k-1)\\
x_{AM_2}(k)&=&  4x_{AM_1}(k-1) \oplus  3x_{BM_2}(k-1)\\
x_{BM_1}(k)&=&  4x_{AM_1}(k-1) \oplus  3x_{BM_2}(k-1)\\
x_{BM_2}(k)&=& 10x_{BM_2}(k-1) \oplus 11x_{AM_1}(k-1)\\
\end{array}
\right.
\end{align*}

\begin{align*}%\label{equ:ordo4}
\left\lbrace
\begin{array}{lcl}
x_{AM_1}(k)&=& 20x_{AM_1}(k-1) \oplus 19x_{BM_2}(k-1) \oplus 15x_{AM2}(k-1)\\
x_{AM_2}(k)&=& 14x_{AM_1}(k-1) \oplus 13x_{BM_2}(k-1) \oplus  9x_{AM2}(k-1)\\
x_{BM_1}(k)&=&  4x_{AM_1}(k-1) \oplus  3x_{BM_2}(k-1)\\
x_{BM_2}(k)&=& 10x_{BM_2}(k-1) \oplus 11x_{AM_1}(k-1) \oplus  6x_{AM2}(k-1)\\
\end{array}
\right.
\end{align*}
