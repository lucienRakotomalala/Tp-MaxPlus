\chapter{Optimisation d'une ressource dans un problème de jobshop}
Cette fois ci, nous allons nous concentrer sur la modélisation d'un problème à base de \emph{jobshop} dans lequel nous étudierons le travail entre deux machines $M_1$ et $M_2$ sur deux pièces $A$ et $B$. Dans un premier temps, nous modéliserons comme dans le chapitre précédent, le GET associé à notre système. Puis nous étudierons chaque marquage initial de notre graphe, selon les ordonnancement possible, afin de déterminer le cycle de temps associé de chacun.

\section{Étude du problème}
%% Ordonnancement 1	:		M2 --> A		B
%%							M1 -->   A - B

%% Ordonnancement 2	:		M2 -->   B - A		NON RESPECT
%%							M1 -->   A - B		NON RESPECT

%% Ordonnancement 3	:		M2 --> 	 A - B
%%							M1 --> 	 B - A

%% Ordonnancement 4	:		M2 --> 	  B	- A
%%							M1 --> B 		A
Le problème de cet exercice peut être décomposé en deux sous problèmes équivalents. Le premier concerne l'ordre dans lequel les pièces $A$ et $B$ peuvent être traités par les machines : $A$ ne peut pas passer sur $M_1$ tant qu'elle n'est pas d'abord passer sur $M_2$. Pour la pièce $B$, elle doit d'abord passer par $M_1$ puis elle va sur $M_2$, comme cela est expliqué dans l'ennoncé. 

Le deuxième problème qui apparait n'en ai pas vraiment un, il s'agit de choisir l'ordonnancement des usinages de pièces. Très naturellement, pour 2 machines et 2 pièces il vient $2\times 2$ possibilité d'ordonnancement qui sont : \begin{enumerate}
\item \label{item:o1}\textbf{Ordonnancement 1} $M_2 \rightarrow A$ puis $M_1 \rightarrow A \rightarrow B$ et  $M_2 \rightarrow B$ 
\item \label{item:o2}\textbf{Ordonnancement 2} $M_2 \rightarrow B \rightarrow A $ parallèle à $M1 \rightarrow A \rightarrow B$
\item \label{item:o3}\textbf{Ordonnancement 3}	$M_2 \rightarrow A \rightarrow B $ parallèle à $M1 \rightarrow B \rightarrow A$
\item \label{item:o4}\textbf{Ordonnancement 4}	$M_1 \rightarrow B$ puis $M_2 \rightarrow A \rightarrow B$ et  $M_1 \rightarrow B$
\end{enumerate}

Pour modéliser notre problème sous forme de GET, nous aurons besoin des évènements qui lient les machines et les pièces entre elles. Nous pouvons obtenir ces informations à partir du tableau donné dans le sujet de TP. Ces temps d'attentes seront utilisés dans les temps d'attentes des places du GET pour modéliser le temps que met une pièce $x$ à être usiné par une machine $y$.

\section{Modélisation des GET et représentation d'état et analyse du cycle associé}
\begin{center}
Dans un premier temps, nous avons réalisé le graph des événements temporisés général(voir figure \ref{fig:get}). Ce graph a 4 transitions (en noir). Elles correspondent, en partant de celle en haut à gauche et dans le sens horaire, à \emph{A est usiné dans M1}, \emph{B est usiné dans M1}, \emph{B est usiné dans M2} et à \emph{A est usiné dans M2}. Les annotations \emph{A}, \emph{B}, \emph{M1} et \emph{M2} ne font pas partis du graph, elles font offices de légendes pour la couleurs des éléments du graph.
\includegraphics[width = \textwidth]{./II/images/GET.pdf}
\captionof{figure}{\label{fig:get} Graphe des Événements Temporisé du \emph{jobshop}}
\end{center}
\subsection{Ordonnancement 1}
Nous modélisons maintenant le cas du premier ordonnancement possible décrit rapidement en \ref{item:o1}. Dans ce cycle, la pièce $A$ doit d'abord passer sur la machine $M_2$ puis prendre la machine $M_1$. Ensuite c'est au tour de la pièce de prendre les machines $M_1$ puis $M_2$. Pour que la pièce $A$ occuppent les première utilisation de A 
\begin{figure}[!ht]
\centering
%\includegraphics[width = \textwidth]{./II/images/GET_1.pdf}
\caption{\label{fig:get} Graphe des Événements Temporisé de l'ordonnancement 1 (\ref{item:o1})}
\end{figure}

\begin{align*}
\left\lbrace
\begin{array}{lcl}
x_{AM_1}(k)&=& 10x_{AM_1}(k-1) \oplus 9x_{BM_2}(k-1)\oplus 7x_{BM_1}(k-1)\\
x_{AM_2}(k)&=&4x_{AM_1}(k-1) \oplus 3x_{BM_2}(k-1)\\
x_{BM_1}(k)&=& 11x_{BM_1}(k-1) \oplus 14x_{AM_1}(k-1) \oplus 13x_{BM_2}(k-1)\\
x_{BM_2}(k)&=& 20x_{BM_2}(k-1) \oplus 21x_{AM_1}(k-1) \oplus 18x_{BM_1}(k-1)\\
\end{array}
\right.
\end{align*} Pour pouvoir utiliser les analyse qui donne une représentation d'état suivante : 
\begin{align}\label{equ:ordo1}
X(k) = \begin{pmatrix}
x_{AM1}(k) \\ x_{AM2}(k) \\ x_{BM1}(k)\\ x_{BM2}(k)
\end{pmatrix} 
= \begin{pmatrix}
10 & \epsilon & 7 &9\\
4 & \epsilon & \epsilon & 3\\
14 & \epsilon & 11 & 13\\
21 & \epsilon & 18 & 20
\end{pmatrix}X(k-1) = A_1X(k-1)
\end{align}
Pour trouver le temps de cycle, nous proposons d'analyser le ou les valeurs propres de A. Nous obtenons avec ScicosLab et la fonction \emph{karp} le résultat suivant : \begin{eqnarray*}
\lambda(A_1) = 20
\end{eqnarray*}
OBSERVATION NECESSAIRE !!!!!!! %%% TODO
\subsection{Ordonnancement 2}
L'ordonnancement 2 n'est pas réalisable.
\subsection{Ordonnancement 3}
\begin{figure}[!ht]
\centering
%\includegraphics[width = \textwidth]{./II/images/GET_1.pdf}
\caption{\label{fig:get} Graphe des Événements Temporisé de l'ordonnancement 3 (\ref{item:o3})}
\end{figure}
\begin{align*}%\label{equ:ordo3}
\left\lbrace
\begin{array}{lcl}
x_{AM_1}(k)&=& 10x_{AM_1}(k-1) \oplus  9x_{BM_2}(k-1)\\
x_{AM_2}(k)&=&  4x_{AM_1}(k-1) \oplus  3x_{BM_2}(k-1)\\
x_{BM_1}(k)&=&  4x_{AM_1}(k-1) \oplus  3x_{BM_2}(k-1)\\
x_{BM_2}(k)&=& 10x_{BM_2}(k-1) \oplus 11x_{AM_1}(k-1)\\
\end{array}
\right.
\end{align*}
Ce système d'équation du GET peut être représenté sous forme espace d'état avec le même vecteur que d'état que dans \ref{equ:ordo1}. Nous obtenons :
\begin{align}\label{eqn:eeOrdo3}
X(k) = \begin{pmatrix}
10 & \epsilon &\epsilon & 9\\
4 &\epsilon &\epsilon & 3\\
4 &\epsilon &\epsilon & 3\\
11 &\epsilon &\epsilon & 10\\
\end{pmatrix}X(k-1) = A_2X(k-1)
\end{align} 
Comme pour l'ordonnancement précédent, nous calculons la valeurs propres du système pour le temps de cycle : 
\begin{eqnarray*}
\lambda(A_2) = 11  
\end{eqnarray*}

\subsection{Ordonnancement 4}
\begin{figure}[!ht]
\centering
%\includegraphics[width = \textwidth]{./II/images/GET_1.pdf}
\caption{\label{fig:get} Graphe des Événements Temporisé de l'ordonnancement 4(\ref{item:o4})}
\end{figure}
\begin{align*}%\label{equ:ordo4}
\left\lbrace
\begin{array}{lcl}
x_{AM_1}(k)&=& 20x_{AM_1}(k-1) \oplus 19x_{BM_2}(k-1) \oplus 15x_{AM2}(k-1)\\
x_{AM_2}(k)&=& 14x_{AM_1}(k-1) \oplus 13x_{BM_2}(k-1) \oplus  9x_{AM2}(k-1)\\
x_{BM_1}(k)&=&  4x_{AM_1}(k-1) \oplus  3x_{BM_2}(k-1)\\
x_{BM_2}(k)&=& 10x_{BM_2}(k-1) \oplus 11x_{AM_1}(k-1) \oplus  6x_{AM2}(k-1)\\
\end{array}
\right.
\end{align*}
\begin{align}\label{eqn::ee_Ordo4}
X(k) = \begin{pmatrix}
20&15&\epsilon&19\\
14&8&\epsilon&13\\
4&\epsilon&\epsilon&3\\
11&6&\epsilon&10
\end{pmatrix}X(k-1) = A_3X(k-1)
\end{align}