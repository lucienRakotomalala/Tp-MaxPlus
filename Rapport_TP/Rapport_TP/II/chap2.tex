\chapter{Optimisation d'une ressource dans un problème de jobshop}
Cette fois ci, nous allons nous concentrer sur la modélisation d'un problème à base de \emph{jobshop} dans lequel nous étudierons le ravail entre deux machines $M_1$ et $M_2$ sur deux pièces $A$ et $B$. Dans un premier temps, nous modéliserons comme dans le chapitre précédent le GET associé à notre système. Puis nous étudierons chaque marquage initial de notre graphe, selon les ordonnancement possible, pour déterminer le cycle associé de chacun.

\section{GET associé à chaque solution}

\section{Représentation d'état et analyse du cycle associé}